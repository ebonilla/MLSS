%%%%%%%%%%%%%%%%%%%%%%%%%%%%%%%%%%%%%%%%%%%%%%%%%%%%

{\huge {\bf Technical Program}}\\[5mm]

%%%%%%%%%%%%%%%%%%%%%%%%%%%%%%%%%%%%%%%%%%%%%%%%%%%%


\section*{Overview}
\label{program_overview}

\begin{center}
 %\begin{tabular}{|c|c|c|c|c|}\hline
%\begin{tabular}{|p{3cm}|p{3cm}|p{3cm}|p{3cm}|p{3cm}|}\hline
\begin{tabular}{|x{3cm}|x{3cm}|x{3cm}|x{3cm}|x{3cm}|}\hline
Monday & Tuesday & Wednesday & Thursday & Friday\tabularnewline \hline
\multicolumn{4}{|c|}{8:30-10:00}&\tabularnewline  
\multicolumn{4}{|c|}{Workshops} & \tabularnewline \hline 
\multicolumn{4}{|c|}{10:00-10:30} & \tabularnewline
\multicolumn{4}{|c|}{Coffee Break} &  \tabularnewline\hline
\multicolumn{4}{|c|}{10:30-12:00} & 10:30-12:00  \tabularnewline
\multicolumn{4}{|c|}{Workshops} &  ACFR Lab Tour\tabularnewline\hline 
\multicolumn{3}{|c|}{12:00-13:15} & 12:00-13:00 & \tabularnewline
\multicolumn{3}{|c|}{Lunch} & Lunch  & \tabularnewline\hline
\multicolumn{3}{|c|}{13:15-14:05} & 13:00-13:50 & 13:00-15:00\tabularnewline
\multicolumn{3}{|c|}{Invited Talk} & Invited Talk  &  Fellowship\tabularnewline\hline
\multicolumn{3}{|c|}{14:05-15:20} & 13:50-15:05 & \tabularnewline
\multicolumn{3}{|c|}{Oral Session 1} & {Oral Session 1} & \tabularnewline\hline
\multicolumn{3}{|c|}{15:20-15:45} & 15:05-15:30 & \tabularnewline
\multicolumn{3}{|c|}{Coffee Break} & {Coffee Break} & \tabularnewline\hline
\multicolumn{3}{|c|}{15:45-17:00} & 15:30-17:00 & \tabularnewline
\multicolumn{3}{|c|}{Oral Session 2} & {Oral Session 2} & \tabularnewline\hline
\multicolumn{4}{|c|}{17:00-18:30} & \tabularnewline
\multicolumn{4}{|c|}{Interactive Poster Sessions} & \tabularnewline\hline
& & & 18:30-Late&\tabularnewline
& & & Banquet&\tabularnewline\hline

\end{tabular}

\end{center}




\newpage
\begin{tabular}{lp{13.8cm}}
\hline
\multicolumn{2}{|c|}{{\bf Monday, July 9, 2012}}\\
\hline\\
{\bf 8:30-10:00} & {\bf Workshops} \\[2mm]
& W1: {\em Generating Robot Motion for Contact with the World}\\
& W2: {\em Resource-Efficient Integration of Perception and Control for Highly Dynamic Mobile Systems}\\
& W3: {\em Robots in Clutter: Manipulation, Perception and Navigation in Human Environments}\\
& W4: {\em RGB-D: Advanced Reasoning with Depth Cameras}\\
& W5: {\em Algorithmic Frontiers in Medical Robotics: Coping with Uncertain, Deformable, Heterogenous Environments}\\[2mm]

{\bf 10:00-10:30} & {\bf Coffee Break} \\[4mm]

{\bf 10:30-12:00} & {\bf Workshops Continued} \\[4mm]

{\bf 12:00-13:15} & {\bf Welcome Lunch} \\[4mm]

{\bf 13:15-14:05} & {\bf Invited Talk} \\[2mm]
& \em{The robot and the philosopher: charting progress at the Turing centenary}\\
& Anders Sandberg\\[2mm]

{\bf 14:05-14:30} & {\bf 5 Minute Talks} \\[2mm]
& \em{ Towards Persistent Localization and Mapping with a Continuous Appearance-based Topology}\\
& William Maddern\phantomsection\label{Maddern}, Michael Milford, Gordon Wyeth\\[2mm]
& \em{ Turning-rate Selective Control
: A New Method for Independent Control of Stress-engineered MEMS Microrobots}\\
& Igor Paprotny\phantomsection\label{Paprotny}, Christopher Levey, Bruce Donald\\[2mm]
& \em{ Rigidity Maintenance Control for Multi-Robot Systems}\\
& Daniel Zelazo\phantomsection\label{Zelazo}, Antonio Franchi, Frank Allgöwer, Heinrich Bülthoff, Paolo Robuffo Giordano\\[2mm]
& \em{ State Estimation for Legged Robots - Consistent Fusion of Leg Kinematics and IMU}\\
& Michael Bloesch\phantomsection\label{Bloesch}, Marco Hutter, Mark Hoepflinger, Stefan Leutenegger, Christian Gehring, C. David Remy, Roland Siegwart\\[2mm]
& \em{ Toward Information Theoretic Human-Robot Dialog}\\
& Stefanie Tellex\phantomsection\label{Tellex}, Pratiksha Thaker, Robin Deits, Thomas Kollar, Nicholas Roy\\[2mm]

{\bf 14:30-14:55} & {\bf Award Talk} \\[2mm]
& \em{ Towards A Swarm of Agile Micro Quadrotors}\\
& Aleksandr Kushleyev\phantomsection\label{Kushleyev}, Vijay Kumar, Daniel Mellinger\\[2mm]
\end{tabular}

\newpage
\begin{tabular}{lp{13.8cm}}
{\bf 14:55-15:20} & {\bf 5 Minute Talks} \\[2mm]
& \em{ Exploiting Passive Dynamics with Variable Stiffness Actuation in Robot Brachiation}\\
& Jun Nakanishi\phantomsection\label{Nakanishi}, Sethu Vijayakumar\\[2mm]
& \em{ Parsing Indoor Scenes Using RGB-D Imagery}\\
& Camillo Taylor\phantomsection\label{Taylor}, Anthony Cowley\\[2mm]
& \em{ Probabilistic Modeling of Human Dynamics for Intention Inference}\\
& Zhikun Wang\phantomsection\label{Wang}, Marc Deisenroth, Heni Ben Amor, David Vogt, Bernhard Schölkopf, Jan Peters\\[2mm]
& \em{ Efficiently Finding Optimal Winding-Constrained Loops in the Plane}\\
& Paul Vernaza\phantomsection\label{Vernaza}, Venkatraman Narayanan\\[2mm]
& \em{ Time-Optimal Trajectory Generation for Path Following with Bounded Acceleration and Velocity}\\
& Tobias Kunz\phantomsection\label{Kunz}, Mike Stilman\\[2mm]

{\bf 15:20-15:45} & {\bf Coffee Break} \\[4mm]

{\bf 15:45-16:10} & {\bf Award Talk} \\[2mm]
& \em{ Affine Trajectory Deformation for Redundant Manipulators}\\
& Quang-Cuong Pham\phantomsection\label{Pham}, Yoshihiko Nakamura\\[2mm]

{\bf 16:10-16:35} & {\bf 5 Minute Talks} \\[2mm]
& \em{ Robust Object Grasping using Force Compliant Motion Primitives}\\
& Moslem Kazemi\phantomsection\label{Kazemi}, Jean-Sebastien Valois, J. Andrew Bagnell, Nancy Pollard\\[2mm]
& \em{ Multi-Stage Micro Rockets for Robotic Insects}\\
& Mirko Kovac\phantomsection\label{Kovac}, Rohit Krishnan, Maria Bendana, Jessica Burton, Michael Smith, Robert Wood\\[2mm]
& \em{ Extrinsic Calibration from Per-Sensor Egomotion}\\
& Jonathan Brookshire\phantomsection\label{Brookshire}, Seth Teller\\[2mm]
& \em{ Probabilistic Temporal Logic for Motion Planning with Resource Threshold Constraints}\\
& Chanyeol Yoo\phantomsection\label{Yoo}, Robert Fitch, Salah Sukkarieh\\[2mm]


{\bf 16:35-17:00} & {\bf Early Career Spotlight} \\[2mm]
& \em{Towards Motor Skill Learning for Robotics}\\
& Jan Peters\\[2mm]

{\bf 17:00-18:30} & {\bf Interactive Poster Session} \\[2mm]
\end{tabular}


\newpage
\begin{tabular}{lp{13.8cm}}
\hline
\multicolumn{2}{|c|}{{\bf Tuesday, July 10, 2012}}\\
\hline\\
{\bf 8:30-10:00} & {\bf Workshops} \\[2mm]
& W1: {\em Generating Robot Motion for Contact with the World}\\
& W2: {\em Resource-Efficient Integration of Perception and Control for Highly Dynamic Mobile Systems}\\
& W3: {\em Robots in Clutter: Manipulation, Perception and Navigation in Human Environments}\\
& W4: {\em RGB-D: Advanced Reasoning with Depth Cameras}\\
& W5: {\em Algorithmic Frontiers in Medical Robotics: Coping with Uncertain, Deformable, Heterogenous Environments}\\
& T1: {\em Machine Learning for Robotics: Old Dreams, New Tools}\\[2mm]

{\bf 10:00-10:30} & {\bf Coffee Break} \\[4mm]

{\bf 10:30-12:00} & {\bf Workshops Continued} \\[4mm]

{\bf 12:00-13:15} & {\bf Lunch Break} \\[4mm]

{\bf 13:15-14:05} & {\bf Invited Talk} \\[2mm]
& \em{Small Brains, Small Planets}\\
& Mandyam Srinivasan\\[2mm]

{\bf 14:05-14:30} & {\bf 5 Minute Talks} \\[2mm]
& \em{ Recognition and Pose Estimation of Rigid Transparent Objects with a Kinect Sensor }\\
& Ilya Lysenkov\phantomsection\label{Lysenkov}, Victor Eruhimov, Gary Bradski\\[2mm]
& \em{ On the Structure of Nonlinearities in Pose Graph SLAM}\\
& Heng Wang\phantomsection\label{Wang2}, Gibson Hu, Shoudong Huang, Gamini Dissanayake\\[2mm]
& \em{ Hybrid Operational Space Control for Compliant Legged Systems}\\
& Marco Hutter\phantomsection\label{Hutter}, C. David Remy, Mark Hoepflinger, Christian Gehring, Michael Bloesch, Roland Siegwart\\[2mm]
& \em{ Asymptotically-optimal Path Planning on Manifolds}\\
& Leonard Jaillet\phantomsection\label{Jaillet}, Josep Porta\\[2mm]
& \em{ Physics-Based Grasp Planning Through Clutter}\\
& Mehmet Dogar\phantomsection\label{Dogar}, Kaijen Hsiao, Matei Ciocarlie, Siddhartha Srinivasa\\[2mm]

{\bf 14:30-14:55} & {\bf Award Talk} \\[2mm]
& \em{ Formalizing Assistive Teleoperation}\\
& Anca Dragan\phantomsection\label{Dragan}, Siddhartha Srinivasa\\[2mm]
\end{tabular}

\newpage
\begin{tabular}{lp{13.8cm}}
{\bf 14:55-15:20} & {\bf 5 Minute Talks} \\[2mm]
& \em{ The Banana Distribution is Gaussian: A Localization Study with Exponential Coordinates}\\
& Andrew Long\phantomsection\label{Long}, Kevin Wolfe, Michael Mashner, Gregory Chirikjian\\[2mm]
& \em{ Modeling and Prediction of Pedestrian Behavior based on the Sub-goal Concept}\\
& Tetsushi Ikeda\phantomsection\label{Ikeda}, Yoshihiro Chigodo, Daniel Rea, Francesco Zanlungo, Masahiro Shiomi, Takayuki Kanda\\[2mm]
& \em{ Real-Time Inverse Dynamics Learning for Musculoskeletal Robots based on Echo State Gaussian Process Regression}\\
& Christoph Hartmann\phantomsection\label{Hartmann}, Joschka Boedecker, Oliver Obst, Shuhei Ikemoto, Minoru Asada\\[2mm]
& \em{ M-Width: Stability and Accuracy of Haptic Rendering of Virtual Mass}\\
& Nick Colonnese\phantomsection\label{Colonnese}, Allison Okamura\\[2mm]

{\bf 15:20-15:45} & {\bf Coffee Break} \\[4mm]

{\bf 15:45-16:10} & {\bf Award Talk} \\[2mm]
& \em{ Walking and Running on Yielding and Fluidizing Ground}\\
& Feifei Qian\phantomsection\label{Qian}, Tingnan Zhang, Chen Li, Aaron Hoover, Pierangelo Masarati, Paul Birkmeyer, Andrew Pullin, Ronald Fearing, Dan Goldman\\[2mm]

{\bf 16:10-16:35} & {\bf 5 Minute Talks} \\[2mm]
& \em{ Nonparametric Bayesian Models for Unsupervised Scene Analysis and Reconstruction}\\
& Dominik Joho\phantomsection\label{Joho}, Gian Diego Tipaldi, Nikolas Engelhard, Cyrill Stachniss, Wolfram Burgard\\[2mm]
& \em{ A Distributable and Computation-flexible Assignment Algorithm: From Local Task Swapping to Global Optimality}\\
& Lantao Liu\phantomsection\label{Liu}, Dylan Shell\\[2mm]
& \em{ What's in the Bag: A Distributed Approach to D Fabrication by Duplication with Modular Robots}\\
& Kyle Gilpin\phantomsection\label{Gilpin}, Daniela Rus\\[2mm]
& \em{ What Types of Interactions do Bio-Inspired Robot Swarms and Flocks Afford a Human?}\\
& Michael Goodrich\phantomsection\label{Goodrich}, Sean Kerman, Brian Pendleton, P.B. Sujit\\[2mm]



{\bf 16:35-17:00} & {\bf Early Career Spotlight} \\[2mm]
& \em{Towards Lifelong Navigation for Mobile Robots}\\
& Cyrill Stachniss\\[2mm]

{\bf 17:00-18:30} & {\bf Interactive Poster Session} \\[2mm]
\end{tabular}




\newpage
\begin{tabular}{lp{13.8cm}}
\hline
\multicolumn{2}{|c|}{{\bf Wednesday, July 11, 2012}}\\
\hline\\
{\bf 8:30-10:00} & {\bf Workshops} \\[2mm]
& W6: {\em Long-term operation of autonomous robotic systems in changing environment}\\
& W7: {\em Stochastic Motion Planning and Information-Based Control}\\
& W8: {\em Beyond laser and vision: alternative sensing techniques for robotic perception}\\
& W9: {\em Biologically Inspired Robotics}\\
& W10: {\em  Robotics for Environmental Monitoring}\\[2mm]


{\bf 10:00-10:30} & {\bf Coffee Break} \\[4mm]

{\bf 10:30-12:00} & {\bf Workshops Continued} \\[4mm]

{\bf 12:00-13:15} & {\bf Lunch Break} \\[4mm]

{\bf 13:15-14:05} & {\bf Invited Talk} \\[2mm]
& \em{ Single Atom Devices for Quantum Computing}\\
& Michelle Simmons\\[2mm]

{\bf 14:05-14:30} & {\bf 5 Minute Talks} \\[2mm]
& \em{ Robust Navigation Execution by Planning in Belief Space}\\
& Bhaskara Marthi\phantomsection\label{Marthi}\\[2mm]
& \em{ Failure Anticipation in Pursuit-Evasion}\\
& Cyril Robin\phantomsection\label{Robin}, Simon Lacroix\\[2mm]
& \em{ Inference on Networks of Mixtures for Robust Robot Mapping}\\
& Edwin Olson\phantomsection\label{Olson}, Pratik Agarwal\\[2mm]
& \em{ Recognition, Prediction, and Planning for Assisted Teleoperation of Freeform Tasks}\\
& Kris Hauser\phantomsection\label{Hauser}\\[2mm]
& \em{ Hierarchical Motion Planning in Topological Representations}\\
& Dmitry Zarubin\phantomsection\label{Zarubin}, Vladimir Ivan, Marc Toussaint, Taku Komura, Sethu Vijayakumar\\[2mm]

{\bf 14:30-14:55} & {\bf Award Talk} \\[2mm]
& \em{ Visual Route Recognition with a Handful of Bits}\\
& Michael Milford\phantomsection\label{Milford}\\[2mm]
\end{tabular}

\newpage
\begin{tabular}{lp{13.8cm}}
{\bf 14:55-15:20} & {\bf 5 Minute Talks} \\[2mm]
& \em{ CompAct Arm: a Compliant Manipulator with Intrinsic Variable Physical Damping}\\
& Matteo Laffranchi\phantomsection\label{Laffranchi}, Nikos Tsagarakis, Darwin Caldwell\\[2mm]
& \em{ Fast Weighted Exponential Product Rules for Robust Distributed Data Fusion in General Multi-Robot Networks}\\
& Nisar Ahmed\phantomsection\label{Ahmed}, Jonathan Schoenberg, Mark Campbell\\[2mm]
& \em{ Estimating Human Dynamics On-the-fly Using Monocular Video For Pose Estimation}\\
& Priyanshu Agarwal\phantomsection\label{Agarwal}, Suren Kumar, Julian Ryde, Jason Corso, Venkat Krovi\\[2mm]
& \em{ Colour-Consistent Structure-from-Motion Models using Underwater Imagery}\\
& Mitch Bryson\phantomsection\label{Bryson}, Matthew Johnson-Roberson, Oscar Pizarro, Stefan Williams\\[2mm]

{\bf 15:20-15:45} & {\bf Coffee Break} \\[4mm]

{\bf 15:45-16:10} & {\bf Award Talk} \\[2mm]
& \em{ On Stochastic Optimal Control and Reinforcement Learning by Approximate Inference}\\
& Konrad Rawlik\phantomsection\label{Rawlik}, Marc Toussaint, Sethu Vijayakumar\\[2mm]

{\bf 16:10-16:35} & {\bf 5 Minute Talks} \\[2mm]
& \em{ Optimization-Based Estimator Design for Vision-Aided Inertial Navigation}\\
& Mingyang Li\phantomsection\label{Li}, Anastasios Mourikis\\[2mm]
& \em{ Development of a Testbed for Robotic Neuromuscular Controllers}\\
& Alexander Schepelmann\phantomsection\label{Schepelmann}, Hartmut Geyer, Michael Taylor\\[2mm]
& \em{ Distributed Approximation of Joint Measurement Distributions Using Mixtures of Gaussians}\\
& Brian Julian\phantomsection\label{Julian}, Stephen Smith, Daniela Rus\\[2mm]
& \em{ Robust Loop Closing Over Time}\\
& Yasir Latif\phantomsection\label{Latif}, Cesar Cadena Lerma, José Neira, \\[2mm]




{\bf 16:35-17:00} & {\bf Early Career Spotlight} \\[2mm]
& \em{Mobile Manipulation for Healthcare}\\
& Charlie Kemp\\[2mm]

{\bf 17:00-18:30} & {\bf Interactive Poster Session} \\[2mm]
\end{tabular}





\newpage
\begin{tabular}{lp{13.8cm}}
\hline
\multicolumn{2}{|c|}{{\bf Thursday, July 12, 2012}}\\
\hline\\
{\bf 8:30-10:00} & {\bf Workshops} \\[2mm]
& W6: {\em Long-term operation of autonomous robotic systems in changing environment}\\
& W7: {\em Stochastic Motion Planning and Information-Based Control}\\
& W8: {\em Beyond laser and vision: alternative sensing techniques for robotic perception}\\
& W11: {\em From theory to practice of performance comparison and result replications in Robotics Research}\\
& W12: {\em Workshop on Aerial Robotics and the Quadrotor Platform} \\[2mm]

{\bf 10:00-10:30} & {\bf Coffee Break} \\[4mm]

{\bf 10:30-12:00} & {\bf Workshops Continued} \\[4mm]

{\bf 12:00-13:00} & {\bf Lunch Break. Please note that today's invited talk begins 15 minutes earlier than usual, at 13:00.} \\[4mm]

{\bf 13:00-13:50} & {\bf Invited Talk} \\[2mm]
& \em{ Machine Learning as Probabilistic Modeling}\\
& Zoubin Ghahramani\\[2mm]

{\bf 13:50-14:15} & {\bf 5 Minute Talks} \\[2mm]
& \em{ Practical Route Planning Under Delay Uncertainty: Stochastic Shortest Path Queries}\\
& Sejoon Lim\phantomsection\label{Lim}, Christian Sommer, Evdokia Nikolova, Daniela Rus\\[2mm]
& \em{ Optimization of Temporal Dynamics for Adaptive Human-Robot Interaction in Assembly Manufacturing}\\
& Ronald Wilcox\phantomsection\label{Wilcox}, Stefanos Nikolaidis, Julie Shah\\[2mm]
& \em{ Contextual Sequence Prediction with Application to Control Library Optimization}\\
& Debadeepta Dey\phantomsection\label{Dey}, Tian Yu Liu, Martial Hebert, J. Andrew Bagnell\\[2mm]
& \em{ Variational Bayesian Optimization for Runtime Risk-Sensitive Control}\\
& Scott Kuindersma\phantomsection\label{Kuindersma}, Roderic Grupen, Andrew Barto\\[2mm]

{\bf 14:15-14:40} & {\bf Award Talk} \\[2mm]
& \em{ Minimal Coordinate Formulation of Contact Dynamics in Operational Space}\\
& Abhinandan Jain\phantomsection\label{Jain}, Cory Crean, Calvin Kuo, Hubertus von Bremen, Steven Myint\\[2mm]
\end{tabular}

\newpage
\begin{tabular}{lp{13.8cm}}
{\bf 14:40-15:05} & {\bf 5 Minute Talks} \\[2mm]
& \em{ Tendon-Driven Variable Impedance Control Using Reinforcement Learning}\\
& Eric Rombokas\phantomsection\label{Rombokas}, Mark Malhotra, Evangelos Theodorou, Yoky Matsuoka, Emanuel Todorov\\[2mm]
& \em{ An Object Based Approach to Map Human Hand Synergies onto Robotic Hands with Dissimilar Kinematics}\\
& Guido Gioioso\phantomsection\label{Gioioso}, Gionata Salvietti, Monica Malvezzi, Domenico Prattichizzo\\[2mm]
& \em{ Feature-Based Prediction of Trajectories for Socially Compliant Navigation}\\
& Markus Kuderer\phantomsection\label{Kuderer}, Henrik Kretzschmar, Christoph Sprunk, Wolfram Burgard\\[2mm]
& \em{ E-Graphs: Bootstrapping Planning with Experience Graphs}\\
& Michael Phillips\phantomsection\label{Phillips}, Benjamin Cohen, Sachin Chitta, Maxim Likhachev\\[2mm]

{\bf 15:05-15:30} & {\bf Coffee Break} \\[4mm]

{\bf 15:30-15:55} & {\bf Award Talk} \\[2mm]
& \em{ Experiments with Balancing on Irregular Terrains using a Mobile Humanoid Robot}\\
& Luis Sentis\phantomsection\label{Sentis}, Josh Petersen, Roland Philippsen\\[2mm]

{\bf 15:55-16:20} & {\bf 5 Minute Talks} \\[2mm]
& \em{ FFT-based Terrain Segmentation for Underwater Mapping}\\
& Bertrand Douillard\phantomsection\label{Douillard}, Navid Nourani-Vatani, Matthew Johnson-Roberson, Stefan Williams, Chris Roman, Oscar Pizarro, Ian Vaughn, Gabrielle Inglis\\[2mm]
& \em{ Guaranteeing High-Level Behaviors while Exploring Partially Known Maps}\\
& Shahar Sarid\phantomsection\label{Sarid}, Bingxin Xu, Hadas Kress-Gazit\\[2mm]
& \em{ Optimal Control with Weighted Average Costs and Temporal Logic Specifications}\\
& Eric Wolff\phantomsection\label{Wolff}, Ufuk Topcu, Richard Murray\\[2mm]
& \em{ Reducing Conservativeness in Safety Guarantees by Learning Disturbances Online: Iterated Guaranteed Safe Online Learning}\\
& Jeremy Gillula\phantomsection\label{Gillula}, Claire Tomlin\\[2mm]





{\bf 16:20-17:00} & {\bf Invited Talk} \\[2mm]
& \em{ SpaceX: TBD}\\
& Andrew Howard\\[2mm]

{\bf 17:00-18:30} & {\bf Interactive Poster Session} \\[2mm]

\end{tabular}


\newpage
\section*{Workshops}
\label{workshops}
\subsubsection*{W1.  Generating Robot Motion for Contact with the World}

{\bf Dates:} July 9 \& 10

{\bf Room:} The Refectory H113

\bigskip
{\bf Organizers:}

Mihail Pivtoraiko (University of Pennsylvania) \\
Dov Katz (Carnegie Mellon University)\\
Oliver Brock (TU Berlin)

\bigskip
{\bf Description:}
The aim of this proposed workshop is to discuss the state of the art in mobile manipulation research. Robust, reliable mobile manipulation is critical for robotics applications in the home, health care and retail industries. The workshop will focus on research at the intersection of motion generation and manipulation contact.




%%%%%%%%%%%%%%%%%%%%%%%%%%%%%%%%%%%%%%%%%%%%%%%%%%%%
%======================================================================
\subsubsection*{W2.  Resource-Efficient Integration of Perception and Control for Highly Dynamic Mobile Systems}
%======================================================================

{\bf Dates:} July 9 \& 10

{\bf Room:} MacRae Room S418

\bigskip
{\bf Organizers:}

Michael Suppa (DLR – Institute of Robotics and Mechatronics)\\
Darius Burschka (TU Munich)\\
Konstantinos Dalamagkidis (TU Munich)\\
Korbinian Schmid (DLR - Institute of Robotics and Mechatronics)

\bigskip
{\bf Description:}
Subjects of interest include, but are not limited to:
\begin{itemize}
\item Robust and accurate perception from limited sensing on light-weight, resource-limited systems (e.g. fusion approaches with an emphasis on error-tolerance and extension of the dynamic range observable)
\item Planning and control approaches for highly dynamic systems to cope with the disadvantages of limited sensing on small platforms
\item Interaction between internal representation and low-/high-level control for scalable action generation under degrading perceptive conditions
\item Integration of perception with action/reaction approaches towards improved performance and safety in small unmanned systems
\item Collaborative multi-system approaches to perception and perception/action solutions for planning and control of unmanned systems with limited sensing.
\item Novel sensor designs and sensing strategies integrated into resource-limited mobile autonomous systems.

\end{itemize}


%%%%%%%%%%%%%%%%%%%%%%%%%%%%%%%%%%%%%%%%%%%%%%%%%%%%
%======================================================================
\subsubsection*{W3.  Robots in Clutter: Manipulation, Perception and Navigation in Human Environments}
%======================================================================

{\bf Dates:} July 9 \& 10

{\bf Room:} History Room S223

\bigskip
{\bf Organizers:}\\
Mehmet Dogar (Carnegie Mellon University)\\
Siddhartha Srinivasa (Carnegie Mellon University)\\
Greg Hager (Johns Hopkins)\\
Kaijen Hsiao (Willow Garage)\\
Matei Ciocarlie (Willow Garage)

\bigskip
{\bf Description:}
Robots operating in our homes will inevitably be confronted with scenes that are simultaneously congested, unorganized, diverse and complex - or, simply put, cluttered. Clutter is a universal problem and severely affects all robot operations: manipulation, perception, navigation, and sensing. This makes it extremely difficult for a single approach to effectively handle clutter, perhaps explaining why robots (and robotics researchers) often shy away from it. 

This workshop aims to bring researchers from different domains together and promote a discussion about clutter. This will contribute to robotics research in at least two ways. First, it will be a venue for the exchange of strategies, ideas, and algorithms used by individual domains. Second, it will provide an opportunity to discuss system-level approaches where manipulation, perception, and navigation work together. We hope to explore directions that will accelerate the deployment of robots into real human settings performing useful tasks even in the presence of clutter.


%%%%%%%%%%%%%%%%%%%%%%%%%%%%%%%%%%%%%%%%%%%%%%%%%%%%
%======================================================================
\subsubsection*{  W4.   RGB-D: Advanced Reasoning with Depth Cameras}
%======================================================================

{\bf Dates:} July 9 \& 10


{\bf Room:} Oriental Room S204

\bigskip
{\bf Organizers:}

Dieter Fox (University of Washington)\\
Kurt Konolige (Willow Garage Inc.)\\
Jana Kosecka (George Mason University)\\
Xiaofeng Ren (University of Washington)

\bigskip
{\bf Description:}
RGB-D (Kinect-style) cameras provide real-time color and dense depth data through active sensing, combining the strengths of passive cameras and laser rangefinders. At \$150 a piece, affordable RGB-D cameras are quickly being adopted as the de facto device for robot perception. At two previous years' RSS conferences, the RGB-D workshops successfully brought together experts from multiple disciplines for presenting and discussing cutting-edge work. Two years after the Kinect release, this year's RGB-D workshop seeks to continue hosting latest developments in RGB-D perception, leading and facilitating efforts across research topics, and building an emerging research community.
Our workshop welcomes high-quality work on all topics related to robotics and RGB-D. We will particularly promote and encourage contributions on two major directions in RGB-D perception: (1) large-scale 3D mapping, using RGB-D cameras to modeling indoor environments in geometry and color as well as semantic structures; and (2) human-robot interaction, applying RGB-D perception to understand humans and to enable robots to perform tasks together with humans.

\clearpage

%%%%%%%%%%%%%%%%%%%%%%%%%%%%%%%%%%%%%%%%%%%%%%%%%%%%
%======================================================================
\subsubsection*{ W5.   Algorithmic Frontiers in Medical Robotics: Coping with Uncertain, Deformable, Heterogenous Environments}
%======================================================================

{\bf Dates:} July 9 \& 10

{\bf Room:} S421

\bigskip
{\bf Organizers:}

Dmitry Berenson (UC Berkeley)\\
Ron Alterovitz (UNC Chapel Hill)\\
Pieter Abbeel (UC Berkeley)\\
Ken Goldberg (UC Berkeley)

\bigskip
{\bf Description:}
Medical robotics is a rapidly-growing field with new devices and methods emerging from industry, academia, and the medical community. One of the great frontiers in this field lies at the algorithmic level. How can we model and simulate 3D deformable tissues? How do we overcome the uncertainty inherent in surgical interventions? How can we integrate motion planning and control with an intuitive user interface? What advances in hardware can be enhanced with algorithmic components? The purpose of this workshop is to bring together researchers, engineers, and physicians studying the above questions and also to discuss applications of methods in robotics to problems in medicine.


%%%%%%%%%%%%%%%%%%%%%%%%%%%%%%%%%%%%%%%%%%%%%%%%%%%%
%======================================================================
\subsubsection*{W6.   Long-term operation of autonomous robotic systems in changing environment}
%======================================================================

{\bf Dates:} July 11 \& 12

{\bf Room:} The Great Hall

\bigskip
{\bf Organizers:}

Daniel Meyer-Delius (KUKA Laboratories GmbH)\\
Patrick Pfaff (KUKA Laboratories GmbH)\\
Gian Diego Tipaldi (University of Freiburg)

\bigskip
{\bf Description:}
The reliable operation of autonomous systems over extended periods of time has been gaining increasing attention in recent years. It is a key aspect in many research projects and a fundamental requirement for any mobile robotic application in the industry. The aim of this workshop is to bring together researchers from different fields and provide a place for discussing the theoretical and practical challenges associated to the reliable operation of autonomous systems over extended periods of time in changing environment.


%%%%%%%%%%%%%%%%%%%%%%%%%%%%%%%%%%%%%%%%%%%%%%%%%%%%
%======================================================================
\subsubsection*{W7.   Stochastic Motion Planning and Information-Based Control}
%======================================================================

{\bf Dates:} July 11 \& 12

{\bf Room:} Oriental Room S204

\bigskip
{\bf Organizers:}

Mac Schwager (Boston University)\\
Michael Vitus (Stanford)\\
Sertac Karaman (MIT)\\
Claire Tomlin (UC Berlekey)

\bigskip
{\bf Description:}
Two fundamental problems in robotics are (1) planning a motion for a
robot to accomplish a specified task in an uncertain environment, and
(2) controlling a robot so that its motion maximally reduces its
uncertainty about the environment. These two problems are intimately
linked; stochastic motion planners implicitly drive robots to reduce
their uncertainty about the environment to ensure that the goal is
reached. Despite this strong link, research in these two topics has
largely proceeded independently from one another. This workshop will
seek to bring together leading researchers in stochastic motion
planning and in information-based control to fuel an exchange of ideas
between these two communities. We will design a full day program of
invited talks and submitted posters aimed at illuminating synergies
between these problems, and spurring advances in both of them. We
will solicit presentations in current stochastic motion planning (SMP)
research areas, including SMP in unknown or uncertain environments;
SMP formulated as a chance constrained optimization program,
randomized SMP techniques, SMP in non-Gaussian belief spaces,
multi-robot SMP, and SMP applications in robotic navigation, grasping
and surgery. We will also solicit presentations in information-based
control research topics, for example control with mutual information
gradients, information surfing, informative path planning, model
predictive control with entropy-based cost, active sensing, and
multi-robot information-based control. Sessions will include a
dedicated time for discussion among the speakers and the audience,
directed by the organizers to address areas of common interest between
stochastic motion planning and information-based control.


%%%%%%%%%%%%%%%%%%%%%%%%%%%%%%%%%%%%%%%%%%%%%%%%%%%%
%======================================================================
\subsubsection*{ W8.   Beyond laser and vision: alternative sensing techniques for robotic perception}
%======================================================================

{\bf Dates:} July 11 \& 12

{\bf Room:} The Refectory H113

\bigskip
{\bf Organizers:}

Thierry Peynot (ACFR, The University of Sydney)\\
Sildomar Monteiro (ACFR, The University of Sydney)\\
Michel Devy (LAAS-CNRS, France)\\
Alonzo Kelly (Carnegie Mellon University, USA)

\bigskip
{\bf Description:}
Perception based on traditional sensing (a visual camera or a laser range finder) has lead to significant realisations in relatively controlled and restricted situations. However, it has also shown important limitations in challenging environments: for example, in the context of field robotics, the presence of airborne dust, smoke, heavy rain or thick fog. As a result, despite significant progress over the last decade, perception arguably remains the bottleneck of greater achievements in robotics.

The future of robotic perception lies in two key elements: “alternative” sensing and intelligent combination of multiple sensing modalities. Robots have an advantage over humans that has been under-exploited so far: they can also sense the environment at various electromagnetic frequencies outside of the visible spectrum, i.e. through alternative sensing modalities. These sensor modalities have opened a range of new possibilities. Examples include: automatic geological analysis using hyperspectral cameras, perception through smoke with infrared imaging, obstacle detection in a dust storm using mm-wave radars or even people detection through walls thanks to Ultra-Wideband radars. Furthermore, by combining data from different sensing modalities that include such alternative sensors, richer environment models can be obtained and higher perception integrity may be achieved.
The main purpose of this workshop is to explore and discuss how alternative sensing and original combinations of sensor data: induce new challenges and perspectives, yield to rethinking conventional perception and data fusion algorithms, open a new range of robotic applications and put the next great robotic achievements within reach.


%%%%%%%%%%%%%%%%%%%%%%%%%%%%%%%%%%%%%%%%%%%%%%%%%%%%
%======================================================================
\subsubsection*{ W9.   Biologically Inspired Robotics}
%======================================================================

{\bf Dates:} July 11 \& 12

{\bf Room:} McRae Room S418

\bigskip
{\bf Organizers:}
Noah Cowan (Johns Hopkins University)
Soon-Jo Chung (University of Illinois at Urbana-Champaign)
Xinyan Deng (Purdue University)

\bigskip
{\bf Description:}
Animals exploit a wide variety of mechanisms for movement, navigation, control, and learning. And, while some engineering systems outperform most biological systems in certain respects (raw speed for example), there is no doubt that animals dramatically outperform their robotic conterparts in complex environments. This workshop will explore shared principles of robotic and biological sensing, actuation, control, and learning. 

Workshop Activities: The goal of this workshop is to bring together engineers and biologists for a half-day meeting to identify and discuss emerging topics and challenges in both communities. The workshop will include invited presentations by researchers from both communities. In addition there will be a poster session comprising submitted abstracts. Invited speakers will participate in an open, interactive panel discussion, with the goal of formulating a new vision for opportunities at the interface between science and engineering.

%%%%%%%%%%%%%%%%%%%%%%%%%%%%%%%%%%%%%%%%%%%%%%%%%%%%
%======================================================================
\subsubsection*{ W10.   Robotics for Environmental Monitoring}
%======================================================================

{\bf Dates:} July 11

{\bf Room:} History Room S223

\bigskip
{\bf Organizers:}
Ryan N. Smith (Queensland University of Technology)\\
Lino Marques (Institute of Systems and Robotics)\\
Ibrahim Volkan Isler (University of Minnesota)\\
Matthew Dunbabin (CSIRO ICT Centre)

\bigskip
{\bf Description:}
Major advances in robotics have been achieved in recent decades, with robots moving from the common manipulator, fixed on the factory floor, to more flexible and autonomous devices, capable of operating in natural and unstructured environments. Today, robots play a fundamental role as data acquisition tools for studying our planet. Some example applications include ocean floor sampling, tracking of plumes, tracking pollution, and monitoring volcanic activity. Design and implementation of robotic systems for environmental research presents significant challenges to robotics researchers. This workshop will bring together researchers with various backgrounds relevant to this multidisciplinary field of research with the intention of creating collaborative links between these communities through sharing recent results and discussing research directions. Presentations will come from researchers in field robotics, environmental sensing, sensor networks, environmental data processing, low-energy robot design, algorithm design, telemetry, energy harvesting, environmentally constrained path planning, multi-scale sampling, and coordination of heterogeneous systems.


\subsubsection*{ W11.   From theory to practice of performance comparison and result replications in Robotics Research}
%======================================================================

{\bf Dates:} July 12


{\bf Room:} McRae Room S418

\bigskip
{\bf Organizers:}

Fabio Bonsignorio (Univeristy Carlos III de Madrid)\\
Angel P. Del Pobi (Jaume I University)\\
John Hallam (University of Southern Denmark)

\bigskip
{\bf Description:}
In 2008 we proposed a set of general guidelines to improve experimentalmethodology and reporting in robotics in order to facilitate experiment replication and performance evaluation and comparison.
We are now at a point where it is possible to give concrete directions for experiment planning and execution, potentially affecting the content of obtained results, not only their 'production process'.
One of the raised objections is the huge variety of subfields in robotics.
The objective of this workshop will be to identify and outline the common ground between two subareas of research: Visual Servoing and SLAM.
It is time to pass from theory to practice, we will discuss the issues and provide examples.

\subsubsection*{ W12.    Workshop on Aerial Robotics and the Quadrotor Platform}
%======================================================================

{\bf Dates:} July 12


{\bf Room:} Room S421

\bigskip
{\bf Organizers:}

Peter Corke (Queensland University of Technology)\\
Robert Mahony (Australian National University)\\
Roland Siegwart (ETHZ)

\bigskip
{\bf Description:}
This workshop will focus on ``robotics problems'' in the deployment of aerial vehicles.   In particular, the workshop will consider questions of perception, manoeuvrability, autonomy, and physical interaction for aerial robotic vehicles.  
The workshop will feature presentations by key researchers in Europe, America and Australasia that provide a perspective on the most challenging problems that are being considered at the moment in their geographical regions.  In addition, there will be significant time set aside for interactive discussion with opportunity for significant input from the audience to determine the key technological and scientific problems for aerial vehicles that face the robotics community.

\section*{Tutorials}
\label{tutorials}

\subsubsection*{T1. Machine Learning for Robotics: Old Dreams, New Tools}
{\bf Dates:} July 10


{\bf Room:} The Great Hall

\bigskip
{\bf Organizers:}

Prof. Sethu Vijayakumar (University of Edinburgh)\\
Prof. Marc Toussaint (FU Berlin)

\bigskip
{\bf Description:}
This tutorial will introduce novel approaches to solving classical problems in robot planning and adaptive control by exploiting the recent advances in statistical machine learning. We will present inference planning methods capable of incorporating uncertainty in a natural way, working at multiple hierarchies and most importantly, bringing to bear the methods and guarantees of probabilistic inference techniques to the domain of robot planning.  A unique interpretation of the stochastic optimal control  formulation will aim to bridge the gap between inference planning techniques, path integral methods and reinforcement learning. We will look at how some of these planning methods can be used effectively to exploit additional redundancies in the emerging field of variable impedance actuation. Finally, we will look at the inner workings of one of the most successful online, incremental learning algorithms capable of working with high dimensional data -- exploring its use for on the fly adaptation of dynamics in the context of Model Predictive Control.

\newpage
\section*{Invited Speakers}
\label{invited_talks}

Monday through Thursday, at the begining of the first oral session. On Thursday, also at the end of the second session.

\subsection*{Anders Sandberg}

\begin{wrapfigure}{r}{2in}
\centering
\includegraphics[width=1.8in]{speakers/andreas.jpg}
\end{wrapfigure}

Anders Sandberg has a Ph.D. in computational neuroscience from Stockholm University. He is currently James Martin Research Fellow at the Future of Humanity Institute at Oxford University. His research at FHI centres on societal and ethical issues surrounding human enhancement and new technology, estimating the capabilities and underlying science of future technologies, as well as issues surrounding global catastrophic risks. Topics of particular interest include enhancement of cognition, cognitive biases, artificial intelligence, neuroethics, rationality, robust reasoning, and public policy. He is also an associate of the Programme on the Impacts of Future Technology, the Oxford Uehiro Center for Practical Ethics and the Oxford Centre for Neuroethics, as well as co-founder of the Swedish think-tank Eudoxa.

\bigskip
{\bf Talk title:} The robot and the philosopher: charting progress at the Turing centenary

\subsection*{Mandyam Srinivasan}
\begin{wrapfigure}{r}{2in}
\centering
\includegraphics[width=1.8in]{speakers/Srinivasan.jpg}
\end{wrapfigure}Srinivasan's research focuses on the principles of visual processing, perception and cognition in
simple natural systems, and on the application of these principles to machine vision and robotics.
He holds an undergraduate degree in Electrical Engineering from Bangalore University, a Master's
degree in Electronics from the Indian Institute of Science, a Ph.D. in Engineering and Applied
Science from Yale University, a D.Sc. in Neuroethology from the Australian National University,
and an Honorary Doctorate from the University of Zurich. Srinivasan is presently Professor of
Visual Neuroscience at the Queensland Brain Institute and the School of Information Technology
and Electrical Engineering of the University of Queensland. Among his awards are Fellowships
of the Australian Academy of Science, of the Royal Society of London, and of the Academy of
Sciences for the Developing World, the 2006 Australia Prime Minister’s Science Prize, the 2008
U.K. Rank Prize for Optoelectronics, the 2009 Distinguished Alumni Award of the Indian Institute
of Science, and the Membership of the Order of Australia (AM).

\bigskip
{\bf Talk title:} Small Brains, Small Planets


\newpage

\subsection*{Michelle Simmons}
\begin{wrapfigure}{r}{2in}
\centering
\includegraphics[width=1.8in]{speakers/Michelle.JPG}
\end{wrapfigure}Professor Simmons is the Director of the Australian Research Council Centre of Excellence for
Quantum Computation and Communication Technology, a Federation Fellow and a Scientia
Professor of Physics at the University of New South Wales. Following her PhD in solar engineering
at the University of Durham in the UK she became a Research Fellow at the Cavendish Laboratory in
Cambridge, UK, working with Professor Sir Michael Pepper FRS in quantum electronics. In 1999, she
was awarded a QEII Fellowship and came to Australia where she was a founding member, and now
the Director of the Centre of Excellence.

Since 2000 she has established a large research group dedicated to the fabrication of atomic-scale
devices in silicon using the atomic precision of a scanning tunneling microscopy. Her group has
developed the world's thinnest conducting wires in silicon and the smallest transistors made with
atomic precision. She has published more than 300 papers in refereed journals and presented over
80 invited and plenary presentations at international conferences. In 2005 she was awarded the
Pawsey Medal by the Australian Academy of Science and in 2006 became the one of the youngest
elected Fellows of this Academy. In 2008 she was awarded a second Federation Fellowship by the
Australian Government and was named the NSW Scientist of the Year in 2011.

\bigskip
{\bf Talk title:} Single Atom Devices for Quantum Computing

\subsection*{Zoubin Ghahramani}
\begin{wrapfigure}{r}{2in}
\centering
\includegraphics[width=1.8in]{speakers/zoubin.jpg}
\end{wrapfigure}

My early childhood was spent in the former Soviet Union and Iran. My family then moved to Spain where I attended the American School of Madrid for 10 years. I studied at the University of Pennsylvania where I was given the Dean's Scholar Award and obtained a BA degree in Cognitive Science and a BSEng degree in Computer Science and Engineering in 1990. In 1995, I obtained my PhD in Cognitive Neuroscience from the Massachusetts Institute of Technology funded by a Fellowship from the McDonnell-Pew Foundation. My dissertation was entitled ``Computation and Psychophysics of Sensorimotor Integration'' and my PhD advisor was Michael Jordan. I moved to the University of Toronto in 1995 where I was an ITRC Postdoctoral Fellow in the Artificial Intelligence Lab of the Department of Computer Science, working with Geoffrey Hinton. From 1998 to 2005, I was faculty at the Gatsby Computational Neuroscience Unit, University College London.

I am currently Professor of Information Engineering, at the University of Cambridge, where I lead the activities in the Machine Learning Group and coordinate Cognitive Systems Engineering. I also have an appointment as Associate Research Professor in the School of Computer Science at Carnegie Mellon University, and I am Adjuct Faculty at the Gatsby Unit, University College London and at POSTECH, South Korea.

My current research interests include Bayesian approaches to machine learning, artificial intelligence, statistics, information retrieval, bioinformatics, and computational motor control. Statistics provides the mathematical foundations for handling uncertainty, making decisions, and designing learning systems. I have recently worked on Gaussian processes, non-parametric Bayesian methods, clustering, approximate inference algorithms, graphical models, Monte Carlo methods, and semi-supervised learning.

\bigskip
{\bf Talk title:} Machine Learning as Probabilistic Modeling

\newpage
\subsection*{Andrew Howard}
\begin{wrapfigure}{r}{2in}
\centering
\includegraphics[width=1.8in]{speakers/howard.jpg}
\end{wrapfigure}

Dr. Andrew Howard is a Research Assistant Professor in the Computer Science Department at the University of Southern California (USC); he is a member of Robotics Research Laboratory (RRL) and the Centre for Robotics and Embedded Systems (CRES). His research interests include multi-robot localization, exploration and coordination, distributed sensor/actuator networks, and simulation of large-scale multi-robot systems. Dr. Howard received his Ph.D. in Engineering from the University of Melbourne in 1999, and his B.Sc. (Hons.) in Theoretical Physics from the same institution in 1991. He served as RoboCup coordinator at the University of Melbourne prior to joining USC in the Fall of 2000.

\bigskip
{\bf Talk title: } SpaceX: TBD

\bigskip

\bigskip

\bigskip

\bigskip

\bigskip

\bigskip


\bigskip





\section*{Early Career Spotlights}
\label{career_talks}

Monday through Wednesday, at the end of the second oral session.

 \subsection*{Jan Peters}
\begin{wrapfigure}{r}{2in}
\centering
\includegraphics[width=2in]{speakers/jan.png}
\end{wrapfigure}Jan Peters is a full professor at the Technische Universitaet Darmstad and a senior research scientist at the Max Planck Institute for Intelligent Systems (MPI-IS) heading the interdepartmental robot learning group. Until 2011, he was was senior research scientist at the Dept. for Empirical Inference and Machine Learning of the Max Planck Institute for Biological Cybernetics (MPI-KYB) in Tuebingen, Germany. He graduated from University of Southern California (USC) with a Ph.D. in Computer Science. He has completed masters degrees in Electrical Engineering (Dipl.-Ing./TU Muenchen), Informatics (Dipl-Inform./FernUni Hagen), Computer Science (M.Sc./USC) and Mechanical Engineering (M.Sc./USC). Jan Peters has been a visiting researcher at the Department of Robotics at the German Aerospace Research Center (DLR) in Oberpfaffenhofen, Germany, at Siemens Advanced Engineering (SAE) in Singapore, at the National University of Singapore (NUS), and at the Department of Humanoid Robotics and Computational Neuroscience at the Advanded Telecommunication Research (ATR) Center in Kyoto, Japan. His research interests include robotics, nonlinear control, machine learning, reinforcement learning, and motor skill learning.

\bigskip
{\bf Talk title:} Towards Motor Skill Learning for Robotics


\newpage
\subsection*{Charlie Kemp}
\begin{wrapfigure}{r}{2in}
\centering
\includegraphics[width=1.5in]{speakers/charlie.jpg}
\end{wrapfigure}
Charles C. Kemp (Charlie) is an Assistant Professor at the Georgia Institute of Technology in the Department of Biomedical Engineering. He is also an Adjunct Assistant Professor in the School of Interactive Computing and the School of Electrical and Computer Engineering. He received a doctorate in Electrical Engineering and Computer Science from MIT in 2005, and his BS and MEng from MIT. In 2007, he founded his lab, the Healthcare Robotics Lab at Georgia Tech (http://healthcare-robotics.com). His research focuses on mobile manipulation and human-robot interaction with an emphasis on robots for healthcare. He is an active member of the Center for Robotics and Intelligent Machines (RIM@GT) and Georgia Tech's multi-disciplinary Robotics Ph.D. program. He has received the 3M Non-tenured Faculty Award, the Georgia Tech Research Corporation Robotics Award, and the NSF CAREER award. His research has been covered extensively by the popular media, including the New York Times, Technology Review, ABC, and CNN.
 

\bigskip
{\bf Talk title:} Mobile Manipulation for Healthcare

\subsection*{Cyrill Stachniss}
\begin{wrapfigure}{r}{2in}
\centering
\includegraphics[width=1.5in]{speakers/cyrill.jpg}
\end{wrapfigure}

Cyrill Stachniss is a lecturer at the University of Freiburg in Germany. In 2009, he received his habilitation and venia legendi and also served as a guest lecturer at the University of Zaragoza.  Before, he was a postdoc at Freiburg University and a senior researcher at the Swiss Federal Institute of Technology in the Autonomous Systems Lab of Roland Siegwart. In 2006, he finished his PhD thesis entitled ``Exploration and Mapping with Mobile Robots'', supervised by Wolfram Burgard, at the University of Freiburg. Since 2008, he is an associate editor of the IEEE Transactions on Robotics and since 2010 a Microsoft Research Faculty Fellow. In his research, he focuses on probabilistic techniques in the context of mobile robotics, perception, and navigation problems.

\bigskip
{\bf Talk title:} Towards Lifelong Navigation for Mobile Robots



